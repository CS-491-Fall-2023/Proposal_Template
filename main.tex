\documentclass{article}

\usepackage{array}
\usepackage{etoolbox}
\usepackage{fancyhdr}
\usepackage{geometry} 
\usepackage{graphicx}
\usepackage{soul}
\usepackage{titling}

%%%%%%%%%%%%%%%%%%%%%%%%%%%%%%%%%%%%%%%%%%%%%%%%%%%%%%%%%%%%
% BEGIN METADATA: Edit the following as appropriate
%%%%%%%%%%%%%%%%%%%%%%%%%%%%%%%%%%%%%%%%%%%%%%%%%%%%%%%%%%%%

\title{Project Title}  % the title of your project
\newcommand\shorttitle{\thetitle}  % if needed: a shorter title for the document header
% Team members.
\newcommand\firstname{Student 1}  % full name
\newcommand\firstid{id01}         % ID, e.g. xy01234
\newcommand\secondname{Student 2} % full name
\newcommand\secondid{id02}        % ID, e.g. xy01234
\newcommand\thirdname{Student 3}  % full name
\newcommand\thirdid{id03}         % ID, e.g. xy01234
% Uncomment the rows for the next 2 students if and as needed.
% \newcommand\fourthname{Student 4} % full name
% \newcommand\fourthid{id04}        % ID, e.g. xy01234
% \newcommand\fifthname{Student 5}  % full name
% \newcommand\fifthid{id05}         % ID, e.g. xy01234

%%%%%%%%%%%%%%%%%%%%%%%%%%%%%%%%%%%%%%%%%%%%%%%%%%%%%%%%%%%%
% END METADATA: Do not edit the preamble any further.
%%%%%%%%%%%%%%%%%%%%%%%%%%%%%%%%%%%%%%%%%%%%%%%%%%%%%%%%%%%%

\pagestyle{fancy}
\lhead{Kaavish Proposal}
\chead{\shorttitle}
\rhead{Fall 2023}
\cfoot{Page \thepage}
\renewcommand{\footrulewidth}{0.4pt}

\newcommand\instruction[1]{\textit{#1}}

\begin{document}

% Cover page.
\begin{titlepage}

\center % Center everything on the page
 
%----------------------------------------------------------------------------------------
%	HEADING SECTIONS
%----------------------------------------------------------------------------------------

\textsc{
  {\LARGE \bf \thetitle}\\\bigskip\bigskip % Your Project Title
  {\large
    Kaavish Project Proposal\\\bigskip
    By}
}\\\bigskip 

%----------------------------------------------------------------------------------------
%	AUTHOR SECTION
%----------------------------------------------------------------------------------------

{\large
  \begin{tabular}{ll}
    \firstname & (\firstid@st.habib.edu.pk) \\
    \secondname & (\secondid@st.habib.edu.pk) \\
    \thirdname & (\thirdid@st.habib.edu.pk) \\
    \ifdef{\fourthname}{\fourthname & (\fourthid@st.habib.edu.pk) \\}{}
    \ifdef{\fifthname}{\fifthname & (\fifthid@st.habib.edu.pk) \\}{}
  \end{tabular}
}
\bigskip\bigskip\bigskip

{\large \today}\\\bigskip\bigskip

\includegraphics[height=5cm]{HU_logo}\\\bigskip
 
%----------------------------------------------------------------------------------------
{\large
  In partial fulfillment of the requirement for \\\medskip
Bachelor of Science \\\medskip
Computer Science
}\\\bigskip\bigskip\bigskip

{\large
  \textsc{
    Dhanani School of Science and Engineering\\\bigskip
    Habib University\\\bigskip 
    Fall 2023
  }\\\bigskip\bigskip 
  Copyright @ 2023 Habib University
}

\end{titlepage}


%%%%%%%%%%%%%%%%%%%%%%%%%%%%%%%%%%%%%%%%%%%%%%%%%%%%%%%%%%%%
% DATA: Populate the rest of the document as instructed.
%%%%%%%%%%%%%%%%%%%%%%%%%%%%%%%%%%%%%%%%%%%%%%%%%%%%%%%%%%%%
\section{Abstract}
\instruction{Please write a 500-600 word abstract on the project idea. It should not be very technically written but should be understandable by anyone.}

\section{Problem definition}
\instruction{Describe the problem that the project addresses.}

\section{Social relevance}
\instruction{Describe any societal problem that the project addresses.}

\section{Originality/Novelty}
\instruction{Describe the value of solving the problem. Compare and contrast with any existing solutions.}

\section{CS contribution}
\instruction{Describe the CS component of the project, e.g. the higher level CS courses that contribute to it.}

\section{Scope and Deliverables}
\instruction{Justify the scope of the project with respect to the size of the team and the year long duration. List the foreseeable deliverables.}

\section{Feasibility}
\instruction{List the resources, e.g. datasets, compute resources, software libraries, hardware, required for the project. Mention how you expect to access and utilize them for the project.}

\section{Team dynamics}
\instruction{Justify the suitability of the team members to the project. For example, their relevant courses, projects, internships, or research.}

\section{Tech stack}
\instruction{Write details of the tech stack you will use for this project for e.g. if you are using MERN stack, you can write MongoDB, Express, React and NodeJS etc.}

\section{References}
\instruction{List your references.}

% External advisor undertaking.
\begin{titlepage}

  \newcolumntype{C}[1]{>{\centering\arraybackslash\hspace{0pt}}p{#1}}

  \centerline{\textbf{\ul{Undertaking of Kaavish advisement as an External Supervisor}}}
  \bigskip\bigskip

  I hereby affirm that I have read the project details as described on the preceding pages and agree to undertake advisement of this Kaavish project as an External Supervisor. I understand that this role entails the following.
  \begin{description}
  \item[Meeting] Meeting the project team regularly, at least once every two weeks, for the entire duration of the Kaavish. The meetings may be held remotely if required.
  \item[Advisement] Providing supervision and advice to the team in order to ensure steady progress of the project toward its goals.
  \item[Liaison] Liaising with the Internal Supervisor as required, e.g. to provide feedback or engage in grading.
  \item[Other] Any other task, depending on availability and suitability, relevant to the Kaavish as communicated by the Internal Supervisor or Kaavish Working Group.
  \end{description}

  \bigskip\bigskip\bigskip
  
  \noindent
  \begin{tabular}{@{}lp{.4\textwidth}}
    Name: & \hrulefill\\\\
    Email: & \hrulefill\\\\
    Phone: & \hrulefill\\\\
    Designation: & \hrulefill\\\\
    Affiliation: & \hrulefill\\\\\\
    Signature: & \hrulefill\\
  \end{tabular}
\end{titlepage}


\end{document}

%%% Local Variables:
%%% mode: latex
%%% TeX-master: t
%%% End:
